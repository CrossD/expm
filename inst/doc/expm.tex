\documentclass{article}
  \usepackage{amsmath,url}
  \usepackage[round]{natbib}
  \usepackage[T1]{fontenc}
  \usepackage[english]{babel}
  %\usepackage{lucidabr}
  \usepackage[noae]{Sweave}

  %\VignetteIndexEntry{Using expm in packages}
  %\VignettePackage{expm}

  \title{Using \pkg{expm} in packages}
  \author{Christophe Dutang \\ ENSIMAG, Grenoble INP \\[3ex]
    Vincent Goulet \\ \'Ecole d'actuariat, Universit\'e Laval}
  \date{Jan. 2008 \ {\footnotesize (added note in June 2010)}}

  \newcommand{\code}[1]{\texttt{#1}}
  \newcommand{\proglang}[1]{\textsf{#1}}
  \newcommand{\pkg}[1]{\textbf{#1}}
  \newcommand{\mat}[1]{\mathbf{#1}}

  \bibliographystyle{plainnat}

\begin{document}

\maketitle

\section{Introduction}

The \pkg{expm} package provides an \proglang{R} function \code{expm}
to compute the matrix exponential of a real, square matrix. The matrix
exponential of a matrix $\mat{A}$ is defined as
\begin{align*}
  e^{\mat{A}}
  &= \mat{I} + \mat{A} + \frac{\mat{A}^2}{2!} + \dots \\
  &= \sum_{k = 0}^\infty \frac{\mat{A}^k}{k!}.
\end{align*}

The actual computations are done in \proglang{C} by a function
of the same name that is callable by other packages. Therefore,
package authors can use these functions and avoid duplication of
efforts.

\section{Description of the functions}

The \proglang{R} function \texttt{expm} takes as argument a real,
square matrix and returns its exponential. Dimension names are
preserved:
\begin{Schunk}
\begin{Sinput}
> library(expm)
> m <- matrix(c(4, 1, 1, 2, 4, 1, 0, 1, 4), 3, 3)
> expm(m)
\end{Sinput}
\begin{Soutput}
         [,1]     [,2]      [,3]
[1,] 147.8666 183.7651  71.79703
[2,] 127.7811 183.7651  91.88257
[3,] 127.7811 163.6796 111.96811
\end{Soutput}
\begin{Sinput}
> dimnames(m) <- list(letters[1:3], LETTERS[1:3])
> m
\end{Sinput}
\begin{Soutput}
  A B C
a 4 2 0
b 1 4 1
c 1 1 4
\end{Soutput}
\begin{Sinput}
> expm(m)
\end{Sinput}
\begin{Soutput}
         A        B         C
A 147.8666 183.7651  71.79703
B 127.7811 183.7651  91.88257
C 127.7811 163.6796 111.96811
\end{Soutput}
\end{Schunk}

\bigskip

%% manual centerig of "overlapping" box
\hspace*{-.12\textwidth}% .08 = .16 / 2
\fbox{\begin{minipage}{1.16\textwidth}%% wider than the text!
  Note that the remainder of this text \textbf{mainly} relates to
  \code{expm(., method = "Ward77")}, i.e., the method of \cite{Ward:77} which
  is no longer the default method, as e.g., \code{method = "Higham08"} has
  found to be (``uniformly'') superior, see \cite{Higham:2008}.
\end{minipage}}

\bigskip

The actual computational work is done in \proglang{C} by a routine
defined as
\begin{verbatim}
void expm(double *x, int n, double *z)
\end{verbatim}
where \code{x} is the vector underlying the \proglang{R} matrix and
\code{n} is the number of lines (or columns) of the matrix. The matrix
exponential is returned in \code{z}. The routine uses the algorithm of
\cite{Ward:77} based on diagonal Pad\'e table approximations in
conjunction with three step preconditioning. The Pad\'e approximation to
$e^{\mat{A}}$ is
\begin{displaymath}
  e^{\mat{A}} \approx R(\mat{A}),
\end{displaymath}
with
\begin{align*}
  R_{pq} (\mat{A})
  &= (D_{pq}(\mat{A}))^{-1} N_{pq}(\mat{A}) \\
  \intertext{where}
  D_{pq}(\mat{A})
  &= \sum_{j=1}^p \frac{(p+q-j)! p!}{ (p+q)!j!(p-j)!}\, \mat{A}^j \\
  \intertext{and}
  N_{pq}(\mat{A})
  &= \sum_{j=1}^q \frac{(p+q-j)! q!}{ (p+q)!j!(q-j)!}\, \mat{A}^j.
\end{align*}
See \cite{MolerVanLoan:78} for an exhaustive treatment of the subject.

The \proglang{C} routine is based on a translation made by
\cite{Matrix} of the implementation of the corresponding Octave
function \citep{octave}.


\section{Calling the functions from other packages}

Package authors can use facilities from \pkg{expm} in two (possibly
simultaneous) ways:
\begin{enumerate}
\item call the \proglang{R} level function \code{expm} in
  \proglang{R} code;
\item if matrix exponential calculations are needed in \proglang{C},
  call the routine \code{expm}.
\end{enumerate}

Using \proglang{R} level function \code{expm} in a package simply
requires the following two import directives:
\begin{verbatim}
Imports: expm
\end{verbatim}
in file \code{DESCRIPTION} and
\begin{verbatim}
import(expm)
\end{verbatim}
in file \code{NAMESPACE}.

Accessing the \proglang{C} level routine further requires to prototype
\code{expm} and to retrieve its pointer in the package initialization
function \code{R\_init\_\textit{pkg}}, where \code{\textit{pkg}} is
the name of the package:
\begin{verbatim}
void (*expm)(double *x, int n, double *z);

void R_init_pkg(DllInfo *dll)
{
    expm = (void (*) (double, int, double))  \
            R_GetCCallable("expm", "expm");
}
\end{verbatim}

The definitive reference for these matters remains the \emph{Writing R
Extensions} manual.

\bibliography{expm}

\end{document}

%%% Local Variables:
%%% mode: latex
%%% TeX-master: t
%%% coding: utf-8
%%% End:
